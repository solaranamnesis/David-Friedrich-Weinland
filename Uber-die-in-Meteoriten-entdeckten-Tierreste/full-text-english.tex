\documentclass[a4paper, 12pt, oneside]{article}
\usepackage[utf8]{inputenc}
\usepackage{fouriernc}
\usepackage{booktabs}
\setlength{\emergencystretch}{15pt}
\usepackage{fancyhdr}
\usepackage{graphicx}
\graphicspath{ {./} }
\usepackage{float}
\usepackage{microtype}
\begin{document}
\begin{titlepage} % Suppresses headers and footers on the title page
	\centering % Centre everything on the title page
	\scshape % Use small caps for all text on the title page

	%------------------------------------------------
	%	Title
	%------------------------------------------------
	
	\rule{\textwidth}{1.6pt}\vspace*{-\baselineskip}\vspace*{2pt} % Thick horizontal rule
	\rule{\textwidth}{0.4pt} % Thin horizontal rule
	
	\vspace{1.5\baselineskip} % Whitespace above the title
	
	{\LARGE About the Animal Remains}
	
	\vspace{1.2\baselineskip}
	
	{\LARGE Discovered in the Meteorites}
	
	\vspace{1\baselineskip} % Whitespace above the title

	\rule{\textwidth}{0.4pt}\vspace*{-\baselineskip}\vspace{3.2pt} % Thin horizontal rule
	\rule{\textwidth}{1.6pt} % Thick horizontal rule
	
	\vspace{1\baselineskip} % Whitespace after the title block
	
	%------------------------------------------------
	%	Subtitle
	%------------------------------------------------
	
	{Dr. David Friedrich Weinland} % Subtitle or further description
	
	\vspace*{1\baselineskip} % Whitespace under the subtitle
	
    {\small In commission with G. Fröhner, with Two Woodcuts} % Subtitle or further description
    
	%------------------------------------------------
	%	Editor(s)
	%------------------------------------------------
    \vspace*{\fill}

	{\small\scshape }
	
    Esslingen am Neckar 1882
    
	\vspace{1\baselineskip} % Whitespace above the title

    Internet Archive Online Edition  % Publication year
	
	{\small Attribution NonCommercial ShareAlike 4.0 International } % Publisher
\end{titlepage}
\setlength{\parskip}{1mm plus1mm minus1mm}
\clearpage
\tableofcontents
\clearpage
\section*{Introduction}
\paragraph{}
Shortly before the New Year of 1881, Dr. Otto Hahn in Reutlingen, a lawyer by profession but also an excellent mineralogist and skilled microscopist, wrote a work entitled \emph{The Meteorite (Chondrite) and its Organisms} with thirty-two tables of photographic images (Tübingen, H. Laupp) in which he proves that the meteorites, especially the so-called chondrites, contain organic structures that he, without attempting a thorough and systematic zoological investigation, generally refers to as sponges, corals, and crinoids.

The forms depicted in the above work are purely mechanical, that is, made without the assistance of a draftsman --- and probably every zoologist and paleontologist will obtain the following impression upon examining them: that in large part, if one observes them objectively, \emph{i.e.} without considering their origin, then one involuntarily thinks of organic structures --- because as little as one would like to be inclined to such a presumption at first, and, perhaps due to the highly enthusiastic language and bold conclusions of the text regarding these figures, they seem to demand caution.

Since some of Hahn's images were near to our own interests, because of prior studies of coral made while at sea, we came around to having the relevant cuts transferred for closer inspection. Thereafter, Dr. Hahn provided his entire considerable collection of meteorite cuts, made with great sacrifices of time and money. These cuts, more than six hundred in number, come from eighteen different meteorite falls, mostly duplicates of the Viennese and the extremely rich Tübingen collection. All meteorites are reliably certified and belong to falls from Europe, Asia, and America, some of them from the previous century.

An in-depth study of them this past year has provided the following preliminary results:
\begin{enumerate}
\item The important discovery of Hahn's, great in its consequences, has essentially been confirmed. By far the majority of the forms photographically depicted by Hahn definitely deal with organic remains and have to do with organic structure, indeed, these remains occur in such quantities that some cuts are for the most part composed entirely of them. Well-preserved forms are rare; in the majority it is detritus, large or small, but usually very distinct fragments, the dimensional stability of which can be recognized quite well after one compares many cuts together with the bulk of the material, and as soon as one has familiarized oneself with this strange world of forms, all the more so since individual pieces have been completely preserved or even favorably polished by accident, and can soon provide the best possible way to orient oneself and serve as guiding pieces. However, we expressly state here that the photographic images of Hahn, meritorious as they are, and as much as his above-mentioned work will always remain a foundation, often fail to convey the clarity of the images that we have under the microscope itself.

\item The organic fragments in the chondritic meteorites are firmly caked and sintered together, much like the organic detritus of corals, sponges, mussels, echinoderms, \emph{etc.} in the youngest ocean limestone formations of our Earth. The debris in the meteorites is in fact nothing but petrifacts. The petrifying material is usually, but not always, a silicate often bluish or yellowish in color. Very frequently they contain black, charred, organic masses, that are punctiform or large in extent. In any case, these forms have not experienced a melting process. The melting produced by friction during the passage of the meteorite through the Earth's atmosphere extends, as already shown, only a few millimeters thick over its surface, thus forming the well-known black fusion crust or glaze. The whole interior of the meteorite, at least in the chondritic meteorites, remains untouched.

\item By far the majority of the structures contained in the available meteorites can be subordinated to the classes of polycistines, sponges, and foraminifera, although the types are different from the terrestrial ones.

\item Of coral forms three genera have so far been sufficiently identified, with one perfectly preserved and displaying a fine microscopic structure that one seldom observes in terrestrial fossils. With one exception these corals are among the oldest forms encountered on Earth, the \emph{Favosites}.

\item Of crinoids three forms, but all are still doubtful.

\item We have not been able to detect any trace of the remains of higher animals: mollusks, arthropods, or even vertebrates.

\item Also, plant-based remains have not presently been safely proven. But one often encounters scraps of tissue that could well be plant-based.

\item All the living beings whose remains are embedded in the meteorites we studied, and whose zoological interpretation we have succeeded with thus far, have lived in water and, in accord with their analogously corresponding terrestrial forms, in water that was never allowed to freeze completely.

This situation seems to us to exclude Schiaparelli's recent hypothesis that the meteorites originate from comets or their tails, at least for the chondritic meteorites, provided that stable liquid water on comets cannot be assumed. Or, might the comets themselves partially consist of the remnants of shattered planets? (See also 10 below.)

\item The entire world of forms examined by us in the hundreds of Hahn's cuts, which, based on our preliminary survey and estimation, may well belong to more than fifty different species of living beings, but of which, since they are usually only preserved as broken structural and fragmented pieces, only a minority can be described precisely, and seem to belong to an early evolution of the living world on the celestial body in question, perhaps even antecedent to the oldest fossils in the most senior layers of our Earth.

\item The entire animal world of these meteorites at first gives one the impression of an extraordinary smallness of forms in relation to the terrestrial ones. This impression was already provided by Dr. Hahn and could not be avoided at first. In reality, polyp cups with 0.04 mm diameters in terrestrial corals are not yet known (although there are those with 0.5 mm diameters). But we must not draw any conclusions about the tiny nature of this animal world in comparison with the terrestrial one. The size of the polycistine forms, which we recognized as such (and Hahn was inclined to regard as very small crinoids), as well as the foraminifera, agrees quite well with the terrestrial ones. Moreover, it should be considered that the often difficult-to-interpret structural scraps and tissue meshes of all kinds that appear in the meteorites may very well be the remnants of larger (but probably not higher) life forms. So also in the youngest ocean limestone, as it forms in our tropical sea coasts, where there is found the detritus of crustaceans, echinoderms, corals, polythalamia, \emph{etc.}, with larger and better preserved carapaces \emph{etc.} being always relatively rare while, with the microscope, decipherable structural remains of such occur frequently. However, these are easier to interpret in this case since we can readily examine the associated living forms.

\item The entire world of forms in these meteorites, insofar as we could investigate them, gives the overall impression of a characteristic belonging-together. There are cuts of eighteen different meteorite falls, some from the previous century. The same characteristic forms always return, only more or less frequently. The assumption thus seems to us justified for the time being that all these chondritic meteorites come from a single extraterrestrial celestial body, perhaps a shattered planet, which, in accordance with the analogous construction of its living forms was probably in its physical, and especially in its atmospheric and thermal conditions, not too dissimilar from our Earth.
\end{enumerate}
\paragraph{}
We will now try to briefly characterize some of the most notable genera and species for which there already exists a great deal of material, reserving for later a more comprehensive description with illustrations, especially of the interior structural relations.
\clearpage
\section{``Little Grated Creatures,'' Polycystina}
\subsection{\emph{Phormiscus}. Nov. gen.}
\paragraph{}
($\phi$o$\rho\mu\iota\sigma$xo$\sigma$ = ``little reed basket'')%φορμισxος

Faceted spheres, consisting of glass-clear silica spicules that lay one on top of the other at regular angles like a rush-basket. The spicules are hollow, often furnished with clearly defined longitudinal cavities. Here:
\subsubsection{\emph{Phormiscus vulgaris}. N. sp.}
\paragraph{}
(Image: Hahn, \emph{Meteorite}, Table 29: Figure 2)

Diameter of the whole 0.18 mm. Diameter of the spicule joists 0.05 mm. From the meteorite fall of Knyahinya.

These \emph{Phormiscus} forms are exceptionally common in fragments of the Knyahinya meteorite. There are several types, but the most common one is the one mentioned above, which is immediately recognizable by the thick, clear glass spicule bundles crossed on top of each other at acute angles.
\subsubsection{\emph{Phormiscus grandis}. N. sp.}
\paragraph{}
(Image: Hahn, \emph{Meteorite}, Table 29: Figure 6)

More finely woven than the previous type. The spicules cross at more extensive angles.

The best specimens, which were found later and include the inner structure, are not yet pictured. The diameter of one of such is 3.2 mm. So, it is a big creature that is rather noticeable to the naked eye.

That these \emph{Phormiscus} belong to the Polycistines seems to us certain. The hollow, partially perforated silica spicules, and particularly the spherical shapes, which is conceivable only in animals moving freely in water, points first to this, and not to sponges as one might otherwise think. In any case, however, they form their own family, which we will call Phormiscidae. --- They are certainly not crinoids, as Hahn formerly supposed.
\subsection{\emph{Thyriscus}. Nov. gen.}
\paragraph{}
($\theta\upsilon\rho\iota\sigma$ = ``embrasure'')%θυρις

Similarly faceted spheres, consisting of little silica balls, arranged in such a way that they form quadrangular, inwardly tapering funnels like windows or even better, embrasure constructions. The balls are hollow and often furnished with noticeable perforations. Undoubtedly belongs to the family of Phormiscidae.
\subsubsection{\emph{Thyriscus formosus}. N. sp.}
\paragraph{}
(Image: Hahn, \emph{Meteorite}, Table 30: Figure 3)

The diameter of the whole piece shown here is 0.70 mm. Diameter of an entire funnel 0.35 mm. Diameter of the individual little balls 0.01 mm. Distance of the holes from each other 0.006 mm. Diameter of the holes 0.001 mm. From the meteorite fall of Knyahinya.
\subsection{\emph{Goniobrochus}. Nov. gen.}
\paragraph{}
($\gamma\omega\nu\iota\alpha$ = ``cornered'' and $\beta\rho$o$\chi\sigma$ = ``mesh'')%(γωνια = ``cornered,'' βρὁχος = ``mesh''.)

We establish this genus on very characteristic structural pieces, which occur frequently in our cuts and one of which has been depicted by Hahn in his meteorites on Table 13: Figure 6. It is a tightly assembled, net-like silica tissue intimately grown together, forming an interrelated pane resembling a small silica ball whose cross angles overlap to form almost equilateral, quadrilateral meshes. Where these slats cross, hunches arise like a web of knobs. --- We can also probably place these structures with the Polycistines, among similar skeletal forms depicted by Haeckel in his fine work, \emph{The Radiolarians}, on Table 29. The genera \emph{Stylodictya} and \emph{Stylospira}, which have very similar knob networks forming their inner skeleton, are particularly worthy of consideration. But one might also think of sponges, such as \emph{Scyphia}; or of Bryozoa?
\subsubsection{\emph{Goniobrochus haeckelii}. N. sp.}
\paragraph{}
This form, already depicted by Hahn (see above), comes from the meteorite fall of Cabarras. The available piece appears spread out and fan-shaped in the cut, measuring 0.5 mm crosswise and 0.4 mm in height. The thickness of the little balls is 0.01 mm, the diameter of a stitch is likewise 0.01 mm. The entirety seems to have formed a round pane or perhaps even forming a funnel. We name the species in honor of our former fellow student, the famous founder of the detailed accounts about the great world of these small organisms.
\clearpage
\section{Sponges and Foraminifera}
\subsection*{Family: Uranidae. \emph{Nobis}.}
\paragraph{}
A highly characteristic meteorite type of a lower animal form that occurs very frequently in a wide variety of meteorite falls and, because within the excellent additional cuts we located the finest meteorite form of all --- hardly exempting \emph{Hahnia} (see below) --- can be studied. The same cannot be closely associated with any of the terrestrial animal forms known to us. Whether sponge, whether foraminifera, this question will be difficult to decide, as is well known in some cases of terrestrial fossil forms. Perhaps we are dealing here with an intermediate form.

They are sessile, cushion-shaped colonies with a fine porous lamellar cortex layer and crude, likewise lamellar, lacunae or chambers forming the internal skeleton.
\subsection{\emph{Urania}, Hahn (\emph{sensu stricto}).}
\paragraph{}
We adopt in the strict sense the genus name from Hahn, which he had already established in his work \emph{Primordial Cell}, although as a genus of plants, for this very characteristic meteorite form. Since then, in a number of favorable cuts I have been able to study and draw these interesting forms, which in the Knyahinya meteorite are particularly common, so that any doubt about their animal nature, which Hahn later presumed in his meteorite work, can no longer exist. They are always smalt-blue and cushion-shaped; the very delicate, finely dashed, velvety looking porous cuticle is probably the peduncle of these sessile colonies. In the cross-section one immediately distinguishes a translucent porous cortex layer. The whole interior of the cushion consists of a rather irregular mesh tissue, which radiates from the cortex towards the center smoothing into lamellar lineaments, which have lacuna-like cavities or chambers between them.
\subsubsection{\emph{Urania salve}. N. sp.}
\paragraph{}
This is what we wish to call them, for they are the first greetings of organic forms from another world, the first beings that Hahn recognized as organic, albeit first described as a plant. This species appears as both large and small, as entire individuals and as lots of fragments, it is very common in the meteorites, especially those of Knyahinya. Average size 1 mm. Thickness of the always smalt-blue cortex 0.04 mm. Hahn shows them many times. The large figure of Table 2, all the figures of Table 3: 1, 2, 3, 4, 5, and 6, then Figures 1, 4, and 6 of Table 4, and Figures 1 and 4 of Table 5 also belong here.
\subsection{\emph{Pectiscus}. Nov. gen.}
\paragraph{}
($\pi\eta$x$\tau$o$\sigma$ = ``combed'')%πηxτὁς

Lobate, probably with wide sessile base colonies. They belong to the same family as \emph{Urania}, to the Uranidae. But the cortex layer here is different, coarse, comb-like, \emph{i.e.} formed as stronger more or less radially emanating ribs (lamellae), often reminiscent of the septa of certain coral forms, such as \emph{Fungia}. But the inner structure, however, as we have in several quite excellent cuts before us (see Figure 1, magnified 80 times), consists, as in \emph{Urania}, of a lamellar, chamber-forming tissue that has nothing to do with coral structure. There are a number of species, some of which are apparently quite large, however in the latter only the coarse, inner, chambered mesh-tissue is preserved.
\subsubsection{\emph{Pectiscus zittelii}. N. sp.}
\paragraph{}
The most common species. Based on its external appearance, its radial ribs, and frequently by its overall profile, one is often reminded of the familiar scallops (\emph{Pecten}). But the lobes of these colonies do not maintain a regular overall shape. They are always rounded at the edges; often the edge is divided into smaller lobes by shallow notches. Diameter of the colonies, about 1 to 3 mm. The fine little ribs towards the gray cortex are on average 0.04 mm apart.

Very widespread in the meteorites, particularly those of Knyahinya and of Siena. Also, the large structure to which our \emph{Hahnia} (see below) appears stuck to is such a \emph{Pectiscus}.
\begin{figure}[H]
\centering
\includegraphics[scale=1,keepaspectratio]{fig1.png}
\caption{magnified 80 times}
\end{figure}
\paragraph{}
In Figure 1 we have depicted a small specimen. It comes from the meteorite fall of Iowa [Marion] (February 1847) and indeed provides a clear picture of the internal structure. The outer cortex of the colony at the top and bottom, colored gray here, is preserved. The cut shaved the middle unequally on the two sides; thus, on the lower right one can see the lamellae protruding from the base being quite parallel. In the left half, on the other hand, the cut passed straight through the innermost, mostly irregular, lacuna-like middle layer of the lobe. The entire colony is 1.6 mm long, 1.2 mm wide. --- We have a similar, equally instructive cut from Knyahinya.

We permit ourselves to name this species after the gentleman Professor [Carl Alfred von] Zittel, the thorough researcher of fossil sponges.
\subsubsection{\emph{Pectiscus rudis}. N. sp.}
\paragraph{}
A smaller form with even coarser slats.
\subsection{\emph{Callaion}. Nov. gen.}
\paragraph{}
(x$\alpha\lambda\lambda\alpha\iota$o$\nu$ = ``cockscomb'')%xάλλαιον

One of the most remarkable and beautiful constructions in our meteorite fauna. A fine form, like some sinuate cockscombs, reminiscent of some corals (\emph{Fungia}, \emph{Herpetolithus}) in its striking habitus, but in accordance with the microscopic construction of its cortex layer it might belong with the Uranidae. The thin, outermost layer of the cortex is just as delicately blue-grey, velvety, and even finely striped, as in \emph{Urania}. In most cuts the raised combs that separate the concavities of the colony from each other, as well as in a fine longitudinal cut in which one can recognize these slight depressions, lie beneath the grey cortex tissue composed of parallel or slightly radiating, very regular lamellae, passing through oblique straps connected to each other and located in the innermost structure that, as we know from \emph{Urania} and \emph{Pectiscus}, unfortunately does not show in the best preserved unique specimens, since nowhere does the cut penetrate deep enough. --- In this form we are most vividly reminded of the cross-section of \emph{Carpenteria rhaphidodendron}, a foraminifera of Mauritius, provided by [Carl August] Möbius in his beautiful treatise on the \emph{Eozoön canadense} (\emph{Palaeontology}, 25, Table 40: Figure 60).
\subsubsection{\emph{Callaion paulinianum}. N. sp.}
\paragraph{}
Not shown in Hahn's meteorite atlas.

Widest diameter of the little colonies 2.8 mm, the smallest 2 mm.

It presents itself to the naked eye as a grey, mottled speck. The parallel lamellae, appearing as delicate stripes on the bluish surface, are 0.002 mm apart. The proximal, coarser lamellae 0.01 mm. The individual concavities within the colony sometimes appear as elongated troughs 0.06 mm in diameter, sometimes as roundish, or angular, crater-like depressions from 0.05 to 0.3 mm in diameter. Between these ridges, combs run quite like \emph{Manicina areolata} and many other corals, but of varying width, 0.05 to 0.2 mm in diameter.

The cut comes from the meteorite fall of Iowa. Unfortunately, only one specimen is well preserved, but we also often encountered rudera of this species in meteorite of Knyahinya.

We permit ourselves to name this species in honor of Miss Pauline Schloz, the meritorious sister-in-law of Dr. Hahn, who supported him in the challenging manufacture of the many meteorite cuts with the most self-sacrificing devotion.
\clearpage %only here because of manual fixes
\subsection{\emph{Glossiscus}. Nov. gen.}
\paragraph{}
($\gamma\lambda\omega\sigma\sigma\alpha$ = ``tongue'')%γλῶσσα

Rounded, tongue-like lobe constructing colonies. The cuticle is composed of hexagonal panels. Pores in the recessed furrows and round, recessed holes; no trace of radial ribs as with the Uranidae. Without question belonging to the sponges.
\subsubsection{\emph{Glossiscus schmidtii}. N. sp.}
\paragraph{}
Not pictured by Hahn. On the one on hand, the pores and pore holes of the conspicuously milk-white colored colony appear tinged with black dots, organic matter which has settled in the pores, as is often found in these meteorite fossils. The total length of the lobe is 1.7 mm, the cross-diameter 0.8 mm, the diameter of the pore holes 0.03 to 0.05 mm, the pore furrow 0.02 to 0.04 mm, and the hexagonal panels 0.02 mm.

In a cut of Knyahinya.

We would like to name the species in honor of the famous researcher of living sponges, Professor [Eduard] Oscar Schmidt in Strasbourg.
\subsection{\emph{Carydion}. Nov. gen.}
\paragraph{}
(x$\alpha\rho\upsilon$o$\nu$ = ``nut'')%xάρυον

Glass-clear transparent, like most of these organisms, petrified silica formations that, on average, resemble a nut with a thick carapace and chambers inside. The chambers are created by thick girder constructions, the thick carapace being very porous.

These forms, not depicted by Hahn, are quite common in the meteorites; they are probably sponge-like entities. We just wanted to describe this single species, whose image we will provide later.
\subsubsection{\emph{Carydion solidum}. N. sp.}
\paragraph{}
Diameter of the whole 0.32 mm. The little openings, \emph{i.e.} tubules in the carapace, have a diameter of 0.01 to 0.005 mm. The thickness of the armature forming girders is 0.02 to 0.5 mm. The mesh created by the girders appears three- or four-sided. The thickness of the cortex or carapace is 0.09 mm; the outer contour has entirely rounded corners; the cavities are usually filled with black organic matter. The pores of the cortex are tinged black. The finer structure of the cortex indicates round cells at high magnification. --- From a cut of the Cabarras meteorite fall.
\subsection{\emph{Brochosphaera}. Nov. gen.}
\paragraph{}
($\beta\rho$o$\chi$o$\sigma$ = ``mesh'' and $\sigma\phi\alpha\iota\rho\alpha$ = ``sphere'')%(βρόχος = ``mesh,'' and σφαῖρα = ``sphere'')

Quite common in the meteorites, especially in those of Knyahinya, are fairly extensive coarse-meshed nets, whose wide sutures are composed of more or less distinct, usually hexagonal, cells. Black carbonized particles, of an organic substance, are often attached to the sutures. As a rule, these nets are preserved only as shreds and it was for a long time impossible to obtain an idea of the whole, but finally, in a Knyahinya cut, I encountered an entity that seemed to provide some enlightenment. It is a large, partially cut hemisphere slightly visible to the naked eye, whose outer contours are essentially preserved, and whose interior contains a most beautiful meshwork, as described above. The complete edge of the hemisphere, where it has not been hit by the cut, consists of rather equal hexagonal cells or small panels. The inner space of the hemisphere, which has been exposed by the cut, is traversed by a multi-meshed net whose sutures consist of cells just like those of the exterior.

We can hardly accommodate this structure into any of the known animal groups other than the sponges, but even here it would establish a completely new type. --- None of these forms are pictured by Hahn.
\subsubsection{\emph{Brochosphaera grandis}. N. sp.}
\paragraph{}
Allow us to name this species, of which the best-preserved piece is an available large hemisphere. The diameter of the whole sphere is 3.20 mm. The diameter of the mesh inside is 0.2 to 0.4 mm. The diameter of the frequently elongated, although often quite equilateral, hexagonal cells or little panels that compose the whole is 0.03 to 0.05 mm. The rounded mesh chambers formed by the thick sutures are filled in this available petrifact with a transparent glassy silicate and are often interspersed with lines of fine cracks.

Comes from the meteorite fall of Knyahinya.
\subsubsection{\emph{Brochosphaera hexagonalis}. N. sp.}
\paragraph{}
In this second species, the stated mesh chambers are constantly hexagonal, lying in the mesh as large crystals. A piece of this kind, of which the outer contours are very well preserved, measures 1.20 mm in diameter. The hexagonal, rarely pentagonal, crystal-like meshes are filled with silicates and measure 0.2 mm in diameter; the cells or small panels composing the network are 0.03 to 0.04 mm.

Comes from Knyahinya. There is also a very similar one in a specimen of Cabarras. In another specimen of Knyahinya, the large hexagonal meshes appear regularly in two forms, the majority with 0.26 mm diameter along with a smaller number of ones 0.4 to 0.3 mm in diameter.
\subsection{\emph{Dicheliscus}. Nov. gen.}
\paragraph{}
($\delta\iota\chi\eta\lambda$o$\sigma$ = ``split hoof'')%διχηλος

A striking and characteristic shape, consisting of an interrelated cluster or pane of round bladders. A heavily intruding cut into them allows for some clear insights into their hollow interior. You can see a perpendicular diaphragm going through the middle of the bladder. This separating wall is always thicker on one side than on the other; it arises from a broad base at the end of the cordiform bladder and goes through lamellar-like thinning up to the other end. Such a polished bladder with its diaphragm gives the image of a double split hoof, hence our name: \emph{Dicheliscus}. The fact that the bladders are interrelated with each other seems clear from several parts of the specimen, as we will later depict them.

Until further notice, we would like to initially place these structures with the foraminifera.
\subsubsection{\emph{Dicheliscus uva}. N. sp.}
\paragraph{}
Not shown by Hahn. The diameter of the whole colony is 1.2 mm. Length of the largest cut bladder 0.15 mm. Thickness of the separating wall 0.01 mm. The bladders in the available specimen are of different sizes and all shifts from the grinding are noticeable.

From the Knyahinya meteorite fall.
\subsection{Other forms}
\paragraph{}
Small fragments of regularly winding formations with Polythalamia-like chambers, perhaps belonging to the Rhizopods, have occasionally come to our notice during the inspection of the meteorite cuts. But their preservation is usually not favorable. A fairly pretty piece of this kind, like a small Nautilus, is in a meteorite cut of Cabarras. The total diameter of the little bowl is about 0.5 mm, the chambers 0.05 to 0.1 mm. But these forms require further examination before we dare to determine them.
\clearpage
\section{Corals}
\subsection{\emph{Hahnia}. Nov. gen.}
\paragraph{}
This is the form that, after the strongest doubts, first led me to carry out a more precise zoological study of the entities discovered by Hahn. In fact, its presence alone is decisive. Admittedly, the photographic images of Hahn's, in his meteorite work's Tables 1, 5 and Table 10: Figures 3 and 4, are far from sufficient. A yellow iron staining on the specimen caused quite detrimental black shadows and, in general, microscopic photography has not yet reached the point of reproducing the images with the sharpness that they present to the eye. As valuable as the photographic picture is for larger forms, like the beautiful coral works of Dr. [Carl Benjamin] Klunzinger and [Carl Ludwig] Rominger prove, for the time being, regarding microscopic representation, the hand of the researcher himself, drawing with a full understanding, will not, perhaps ever, be replaced by the mechanical representation. Our \emph{Hahnia}, Figure 2, has unfortunately remained unique to this day. The cut in question belongs to the meteorite fall of Knyahinya. It is one of the most fortunate and also contains very nice scraps of \emph{Urania}, \emph{Pectiscus}, and \emph{Phormiscus}.

Characteristics of the genus \emph{Hahnia}: small microscopic polyp tubes, unequal, large mixed with small, polygonal with rounded corners. The walls of the tubes are thick with sharp linear boundaries towards the outside. At high magnification, a uniformly thick inter-tubular tissue (coenenchyme) becomes visible between the lines bordering the adjacent polyps, which represents a distinct network in the cross-section. Inner longitudinal strips (septa) are missing in the tubes, as well as the transverse dividing walls (tabulae), which are known to divide the individual tubes into floors on top of each other in many similar terrestrial corals. Colony probably encrusted, flat-bottomed, cake shaped.

The genus probably belongs to the Favositidae, a coral family that has long been extinct on Earth, flourishing in the Silurian and Devonian formations, and of which a large number of quite different forms requiring further zoological checks are described in \emph{Paleontology} (Rominger, 1876).
\begin{figure}[H]
\centering
\includegraphics[scale=1,keepaspectratio]{fig2.png}
\caption{\emph{Hahnia meteoritica}, N., attached to a \emph{Pectiscus}. magnified 80 times.}
\end{figure}
\paragraph{}
Diameter of the whole colony 0.90 mm, thus even with the naked eye it can be recognized as a small lentil. Diameter of the individual polyp calyxes 0.04 to 0.1 mm. Diameter of the yellow intermediate pathways, coenenchyme, 0.008 mm. At the corners this becomes swollen, as is often the case with \emph{Favosites}. The striking resemblance of this colony with \emph{Favosites polymorphus} from the Devonian has already been noticed by Professor Quenstedt when Dr. Hahn showed him the object. Even more, it can be compared with \emph{Favosites bimuratus} from the Devonian of Bensberg where the polyp walls and the coenenchyme are remarkably similar, albeit always with the exception of the size ratio. For \emph{Favosites bimuratus} have calyxes measuring from a half to 1 mm.

The individual polyp calyxes in our \emph{Hahnia} are filled with a blackish grey mass, the septa appear greyish white, the coenenchyme yellow. By a lucky coincidence, this coral colony was directly struck from above. In the middle of the picture, the calyxes appear nearly intact; around the edge, particularly on the left side, they are somewhat scuffed, so that one obtains for structural knowledge the very valuable semi-longitudinal cuts through the polyp tubes and can establish the lack of transverse partition walls, as well as of vascular holes (sprout channels).

Hahn's image Table 1: Figure 5 and Table 10: Figure 4 unfortunately is adversely affected by the yellow coloration of the specimen, which becomes black in the photograph.
\subsection{\emph{Calamiscus}. Nov. gen.}
\paragraph{}
(x$\alpha\lambda\alpha\mu\iota\sigma$x$\sigma$ = ``little tubes'')%xαλαμίσxς

\emph{Favosites}-like polyp colonies, consisting of regularly side by side parallel or slightly radial trending, usually glass-clear transparent tubes without longitudinal rails (septa) in the interior, but more or less regularly divided into levels by transverse walls or floors (tabulae) and quite frequently furnished with fine little perforations that mediate the vascular communication between the neighboring tubes. This perfect correspondence of the structure with that of many fossil \emph{Favosites} corals from the Devonian and Silurian formations of the Earth does not make us think of anything other than coral polyps, despite the smallness of the available meteoritic forms. Unfortunately, almost only side cuts are obtained because in this direction the polyp colonies break most easily. In the absence of satisfactory cross-sections, it becomes fairly difficult to distinguish the species of \emph{Calamiscus}; it is left almost exclusively to this: the consistent width of the polyp tubes, the distance of the floors and vascular holes from each other, the horizontal or skewed direction of the floors, and so forth, are purely characteristics that vary quite a bit in one and the same species. --- These entities are exceptionally common in the meteorites, especially in those of Knyahinya.
\subsubsection{\emph{Calamiscus gümbelii}. N. sp.}
\paragraph{}
(Image: Hahn, \emph{Meteorite}, Table 14 and 15)

We base this species on one of the best preserved little colonies in a meteorite cut from the Cabarras fall. It is an oblong, downward pointing colony, as \emph{Favosites} colonies usually are due to the way the species propagates through intermediate grafts shifted down, typical of new tubes. The available colony has a diameter of 0.46 mm and a height of 1 mm, so it is still visible to the naked eye. The diameter of the tubes is 0.01 mm, the distance between the vascular holes, which are exceptionally visible in this polyp colony, from each other is 0.005 to 0.01 mm. The saw-like notch on the side of the tube in Hahn's picture was created by accidental abrasion, in such a way that the funnel-shaped indentation of the little holes comes to light. The floors lay slightly lopsided in the tube, very irregularly spaced from each other, and in general are less common in this colony than in some of the others.

We allow ourselves to name this species after Director Gümbel in Munich, who first subjected the chondritic meteorites to a precise microscopic examination and, in his excellent description of the chondrules in his essay about the stone meteorites found in Bavaria (\emph{Proceedings of the Mathematical and Physical Science Class of the Royal Bavarian Academy of Sciences in Munich}, 1878, p. 14), probably had such \emph{Calamiscus} forms that were less well-preserved but he tried to interpret them mineralogically.
\subsection{\emph{Bosea}. Nov. gen.}
\paragraph{}
One of the most beautiful meteorite structures, without doubt a little bit of a coral colony. A considerable part of the surface, with many distinct larger and smaller little stars, is uniquely preserved. The little stars make up, it would seem, raised flattened little cones; they have up to ten externally broadening septa, separated by dark furrows. The center of the little stars, from which the septa and the furrows emanate, consists of angular granules. The coenenchyme or intermediate area between the little stars appears tiled with angular little plates. Smaller, obviously younger little stars with fewer rays appear between the older ones, such as in an \emph{Astraea}.

I permit myself to designate the genus in honor of Mr. [Carl August] Carl Graf von Bose and Mrs. Louise [Wilhelmine Emilie] Countess von Bose \emph{née} von Reichenbach-Lessonitz, who are both excellent naturalists and took a most active part in these meteorite studies of the author. As is well known, Mrs. Countess von Bose not long ago, through a foundation in Frankfurt am Main, expressed her interest for the exploration of nature in a wonderful way.
\subsubsection{\emph{Bosea cyanea}. Nov. sp.}
\paragraph{}
The above-mentioned colony, everywhere broken off at the margins, has, if it can be obtained, a length of 1.44 mm, a width of 0.88 mm. The diameter of the little stars is 0.04 to 0.08 mm. The diameter of the recessed furrows radiating from the center is 0.003 to 0.006 mm. The petrification material displays the same smalt-blue color as in \emph{Urania salve}. --- This unique piece is in a cut from the fall of Knyahinya.
\clearpage
\section{\emph{Crinoidea}}
\paragraph{}
Our dear friend Dr. Hahn, in Tables 16 thru 30 of his meteorite work, believed that he had to place, for the time being, a large number of forms into this base class of echinoderms. After a more detailed study of their organization, as far as they can be deciphered, we found a number of them more related to the polycistines and sponges, or rather foraminifera. However, there remains a number of forms, which we want to provisionally place with the above animal class, since they cannot be assigned to any other animal type known to us without force and, at least, have certain structural characteristics in common with the crinoids.
\subsection{\emph{Eulophiscus}. Nov. gen.}
\paragraph{}
($\epsilon\upsilon\lambda$o$\phi$o$\sigma$ = ``well-plumed'')%εὔλοφος

A fan-shaped bundle with a central radiating point, undoubtedly floating freely in life, forking at the bottom near the origin once or twice, but no more branches on top of this, rather equal thickness of arms.
\subsubsection{\emph{Eulophiscus quenstedtii}. N. sp.}
\paragraph{}
Here we primarily refer to the pretty picture which Hahn has chosen as the title cover of his meteorite work and displayed smaller in Table 22: Figure 3. However, this object grants a much clearer picture under the microscope than in the photograph. We see five thick arms emanating from the base; the outer left, most favorably situated, shows a cross-section of 0.04 mm at the bottom. Just 0.08 mm above its origin, it bifurcates nicely into two main arms 0.02 mm thick. And they remain equal, as far as one can follow them, which is possible with the one on the left up to the end of the fan, and as far as it is preserved. The aforementioned fork has the form we are accustomed to in the crinoids. But neither here nor in the other arms is a clear crosswise outline visible. It is safe to assume that these arms floated freely in water during life, because you can see them in several places laying down and crossing over each other, hiding under each other, and so forth. The size of the entire tuft is, of course, very minuscule for a crinoid; the height of the whole tuft is only 0.7 mm, the width 1 mm. The whole appears greyish in color, the aforementioned main arms yellowish, semitransparent.

Comes from the fall of Knyahinya.

Perhaps also here are the forms of Hahn, Meteorite, Table 22: Figures 5 and 6.
\subsection{\emph{Euplocamus}. Nov. gen.}
\paragraph{}
($\epsilon\upsilon\pi\lambda$o$\kappa\alpha\mu$o$\sigma$ = ``with goodly locks'')%εὐπλοκαμος

Like one from the previous genus, but in which the arms are not bifurcated.
\subsubsection{\emph{Euplocamus algoideus}. N. sp.}
\paragraph{}
This genus and species are supported for the time being by Hahn's photographs, Table 1: Figure 6, Table 25: Figure 1, and Table 19, all of which represent the same object, and these pictures can be described as quite successful. This pretty piece gives the impression under the microscope of a little tuft of sea algae that has grown on an outcrop of rock. From a patch-shaped constructed central disk, tuft-shaped like the previous, a large number of equally thick arms radiate, which, as far as they are preserved, do not taper. The diameter of the arms is 0.04 mm. The arms are glass-clear transparent. Through the interior of each one runs a dark contour, inferring a fine cavity. Here, too, the arms are laid down and pushing on and over each other, so that one must necessarily think of it as formerly free floating. The whole little stick has a height of 0.8 mm and a width of 1.1 mm, so like the previous one, it is still visible to the naked eye.

Comes from the meteorite fall of Knyahinya.
\subsubsection{\emph{Euplocamus articulatus}. N. sp.}
\paragraph{}
(Image: Hahn, \emph{Meteorite}, Table 23: Figure 4)

A very pretty and distinct object, but less successful in the photographic image. From a base formed by many small, angular plates, a tassel emerges from initially seemingly un-articulated, round, rod-shaped arms, distinguished higher up by clear outline. The structure of this begins in the object with a very marked bend of the arms. These have, as the petrifact clearly indicates, been floating freely through and over each other. The individual arms are round, an inner cavity is not visible, therefore it will probably have to separated later from the genus \emph{Euplocamus}. The diameter of the whole is 1.60 mm. The diameter of the arms under the knee 0.08 mm. At the top, they taper slightly, but only a little. The diameter of the square plates of the base is 0.03 to 0.04 mm. The color of the whole is yellowish, beautiful metallic shiny. --- It is in a cut from the meteorite fall of Knyahinya.
\subsection{\emph{Crobyliscus}. Nov. gen.}
\paragraph{}
(x$\rho\omega\beta\upsilon\lambda$o$\sigma$ = ``knot'')%xρὠβυλος

On a clear one, made of polygonal, mostly hexagonal little plates forming a closed cavity above a number of cylindrical, plait-shaped, tapering towards the end, more massive (not hollow), arm-shaped appendages formed of angular little panes. Is it a crinoid and is that cavity the calyx of it? The fragment upon which we establish this genus is so far a unique piece, whose image we will include in our larger treatise.
\subsubsection{\emph{Crobyliscus fraasii}. N. sp.}
\paragraph{}
Longitudinal diameter of the whole, if obtained, 0.74 mm. Crosswise diameter of the calyx 0.45 mm. Length of the arms, if available, 0.35 mm. Crosswise diameter of the arms 0.3 to 0.6 mm. Thickness of the whorls that comprise the arms, 0.01 to 0.02 mm. Diameter of the angular plates that comprise the calyx, 0.03 to 0.05 mm. The mineral that makes up the structure is undoubtedly silica.

From the meteorite fall of Knyahinya.
\clearpage
\section{Conclusion}
\paragraph{}
With this preliminary characterization of the above sixteen genera of meteorite forms, we believe, for now at least, that we have laid the foundation for a small meteorite fauna. Of all the ones not depicted and in addition to the many already photographically portrayed by Hahn, as far as they are to a lesser extent successful, we will be giving detailed self-drawn images in our larger treatise that is in preparation. These illustrations are already mostly finished.

Regarding the nomenclature of all the new genera established above --- with the exception of \emph{Hahnia} and \emph{Bosea} --- we request, as an authority, to add our name to the name of our dear friend Dr. Hahn, who, though he has taken no direct part in our work, will always remain the one who first asserted the organic origin of these forms and tried to justify them through his ever-valuable atlas and rich collection on which the above work is based.

As we intend to continue these investigations diligently, we would like to conclude with a friendly request to any owners of reliably certified meteorite pieces or cuts to impart them to us for microscopic examination. We will always return them as soon as possible, communicating the results and subsequent public acknowledgments. --- Our address is: Dr. D. F. Weinland, Esslingen, Württemberg.

Printed by E. Blochmann and Sohn in Dresden.
\clearpage
\end{document}
