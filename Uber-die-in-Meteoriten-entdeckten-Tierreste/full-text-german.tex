\documentclass[a4paper, 11pt, oneside]{article}
\usepackage[utf8]{inputenc}
\usepackage[T1]{fontenc}
\usepackage[ngerman]{babel}
\usepackage{fbb} %Derived from Cardo, provides a Bembo-like font family in otf and pfb format plus LaTeX
\usepackage{booktabs}
\setlength{\emergencystretch}{15pt}
\usepackage{fancyhdr}
\usepackage{graphicx}
\graphicspath{ {./} }
\usepackage{float}
\usepackage{microtype}
\usepackage{textalpha}
\begin{document}
\begin{titlepage} % Suppresses headers and footers on the title page
	\centering % Centre everything on the title page
	\scshape % Use small caps for all text on the title page

	%------------------------------------------------
	%	Title
	%------------------------------------------------
	
	\rule{\textwidth}{1.6pt}\vspace*{-\baselineskip}\vspace*{2pt} % Thick horizontal rule
	\rule{\textwidth}{0.4pt} % Thin horizontal rule
	
	\vspace{1.5\baselineskip} % Whitespace above the title
	
	{\LARGE "Uber die in Meteoriten}
	
	\vspace{1.2\baselineskip}
	
	{\LARGE entdeckten Tierreste}
	
	\vspace{1\baselineskip} % Whitespace above the title

	\rule{\textwidth}{0.4pt}\vspace*{-\baselineskip}\vspace{3.2pt} % Thin horizontal rule
	\rule{\textwidth}{1.6pt} % Thick horizontal rule
	
	\vspace{1\baselineskip} % Whitespace after the title block
	
	%------------------------------------------------
	%	Subtitle
	%------------------------------------------------
	
	{Dr. David Friedrich Weinland} % Subtitle or further description
	
	\vspace*{1\baselineskip} % Whitespace under the subtitle
	
    {\small In Kommission bei G. Fr"ohner, mit zwei Holzschnitten\\ Esslingen am Neckar 1882} % Subtitle or further description
    
	%------------------------------------------------
	%	Editor(s)
	%------------------------------------------------
    \vspace*{\fill}

	{\small\scshape }

    Internet Archive Online Edition  % Publication year
	
	{\small Namensnennung Nicht-kommerziell Weitergabe unter gleichen Bedingungen 4.0 International} % Publisher
\end{titlepage}
\setlength{\parskip}{1mm plus1mm minus1mm}
\clearpage
\tableofcontents
\clearpage
\section*{Einleitung}
\paragraph{}
Kurz von Neujahr 1881 hat Herr Dr. O. Hahn in Reutlingen, von Beruf Jurist, dabei aber trefflicher Mineraloge und ge"ubter Mikroskopiker, ein Werk: \emph{Die Meteorite (Chondrite) und ihre Organismen} mit 32 Tafeln photographischer Abbildungen (T"ubingen. H. Laupp) erscheinen lassen, in welchem er den Beweis unternimmt, dass die Meteoriten, besonders speziell die sogenannten Chondrit-Meteoriten, organische Gebilde enthalten, welche er, ohne eine n"ahere, systematisch zoologische Deutung zu versuchen, im Allgemeinen zu den Schw"ammen, Korallen und Crinoiden stellte.

Die in obigem Werke rein mechanisch, also ohne Zutun eines Zeichners, abgebildeten Formen sind --- diesen Eindruck muss wohl jeder Zoologe und Pal"aontologe bei deren Durchmusterung erhalten, --- ihrem gr"o"seren Teile nach solche, bei denen man, wenn man sie unbefangen, d. h. ohne den Gedanken an die Herkunft betrachtet, unwillk"urlich an organische Struktur denken muss, so wenig man zun"achst zu einer solchen Annahme geneigt sein m"ochte und so sehr vielleicht auch der Text zu jenen Abbildungen wegen seiner f"ur die Fachm"anner gar zu begeisterten Sprache und k"uhnen Schl"usse zu kritischer Vorsicht aufzufordern scheinen k"onnte.

Da uns nun einige der Hahn'schen Bilder, wegen fr"uherer Korallenstudien, die wir am Meere gemacht, n"aher interessierten, wandten wir uns an denselben um "uberlassung der betreffenden Schliffe selbst, behufs n"aherer Untersuchung. Darauf hin hat uns Herr Dr. Hahn seine ganze, bedeutende, mit gro"sen Opfern an Zeit und Geld hergestellte Sammlung von Meteoritenschliffen aufs Bereitwilligste zur Verf"ugung gestellt. Diese Schliffe, "uber sechshundert an der Zahl, stammen von achtzehn verschiedenen Meteoritenf"allen, gr"o"stenteils von Dubletten der Wiener und der "uberaus reichen T"ubinger Sammlung. S"amtliche Meteoriten sind sicher beglaubigt und geh"oren F"allen aus Europa, Asien, Amerika, zum Teil solchen aus dem vorigen Jahrhundert an.

Ein in dem letzten Jahre vorgenommenes, eingehendes Studium derselben hat uns nun folgende vorl"aufige Resultate ergeben:
\begin{enumerate}
\item Die wichtige, in ihren Folgen gro"sartige Entdeckung Hahn's hat sich im Wesentlichen best"atigt. Wir haben es bei weitaus der Mehrzahl der von Hahn photographisch abgebildeten Formen ganz entschieden mit organischen Resten, mit organischer Struktur zu tun, ja diese Reste treten teilweise in solcher Menge auf, dass manche Schliffe weitaus der Hauptsache nach ganz aus ihnen zusammengesetzt sind. Gut erhaltene Formen sind selten; der Mehrzahl nach ist es Detritus, es sind gr"o"sere oder kleinere, meist aber sehr deutliche Bruchst"ucke, deren Formbest"andigkeit jedoch bei Vergleich von vielen Schliffen und bei der Masse des Materials, sobald man sich in diese merkw"urdige Formenwelt eingearbeitet hat, recht wohl erkannt werden kann, und dies um so sicherer, als einzelne St"ucke ganz erhalten oder sogar zuf"allig sehr gl"ucklich angeschliffen, uns bald aufs Sch"onste orientieren und als Leitst"ucke dienen k"onnen. wir schicken jedoch hier ausdr"ucklich voraus, dass die photographischen Abbildungen Hahn's, so verdienstlich sie sind und so sehr sein obengenanntes Werk immer eine Grundlage bleiben wird, doch h"aufig nicht die Klarheit der Bilder wiedergeben konnten, die wir unter dem Mikroskop selbst haben.
\item Die organischen Bruchst"ucke sind in den Chondritmeteoriten fest zusammengebacken und zusammengesintert, ganz wie der organische Detritus von Korallen, Schw"ammen, Muscheln, Echinodermen u. s. w. in einer j"ungsten Meereskalkbildung auf unserer Erde. Jene Reste in den Meteoriten sind in der Tat nichts als Petrefakten. Das Versteinerungsmaterial ist in der Regel, wo nicht immer, ein Silikat, "ofters bl"aulich oder gelb gef"arbt. Sehr h"aufig enthalten sie schwarze, verkohlte, organische Masse, punktf"ormig oder in gr"o"serer Ausdehnung. Einen Schmelzungsprocess haben diese Formen jedenfalls nicht durchgemacht. Die Schmelzung, wie sie bei dem Durchgang eines Meteoriten durch die Erdatmosph"are durch Reibung hervorgebracht wird, erstreckt sich, wie der Augenschein schon zeigt, nur auf seine Oberfl"ache und bildet so jene bekannte, schwarze Rinde oder Glasur, die nur eine Dicke von wenigen Millimetern hat. Das ganze Innere des Meteoriten, wenigstens des Chondritmeteoriten, bleibt davon unber"uhrt.
\item Weitaus die Mehrzahl der in den vorliegenden Meteoriten enthaltenen Gebilde l"asst sich den Klassen der Polycistinen, der Schw"amme und der Foraminiferen unterordnen, wenn auch die Typen andere sind als die irdischen.
\item Von Korallenformen konnten bis jetzt drei Gattungen gen"ugend nachgewiesen werden, wovon eine in einem St"uck, das so vollkommen erhalten ist und die feinste mikroskopische Struktur so deutlich zeigt, wie man es bei irdischen Fossilien selten findet. Diese Korallen geh"oren mit Einer Ausnahme zu den "altesten Formen, die uns auf der Erde begegnen, zu den \emph{Favositen}.
\item Von Crinoiden drei Formen, aber alle noch zweifelhaft.
\item Von Resten h"oherer Tiere, von Weichtieren, Gliedertieren oder gar Wirbeltieren haben wir bis jetzt keine Spur entdecken k"onnen.
\item Auch pflanzliche Reste konnten bis jetzt nicht sicher nachgewiesen werden. Doch begegnet man "ofters Gewebefetzen, die wohl pflanzlicher Natur sein k"onnten.
\item Alle Lebewesen, deren Reste in den von uns untersuchten Meteoriten eingebettet sind und deren zoologische Deutung uns bis jetzt gelungen ist, haben im Wasser gelebt und zwar nach Analogie mit den entsprechenden irdischen Formen in einem Wasser, welches nie ganz frieren durfte.

Dieser Umstand scheint uns die neuerdings vielfach angenommene Hypothese von Schiaparelli, dass die Meteoriten den Kometen oder deren Schweifen entstammen, wenigstens f"ur die Chondrit-Meteoriten auszuschlie"sen, sofern konstant fl"ussiges Wasser auf Kometen nicht anzunehmen ist. Oder sollten die Kometen selbst vielleicht teilweise aus Resten zertr"ummerter Planeten bestehen? (Siehe auch unter 10.)
\item Die ganze, von uns in den Hunderten der Hahn'schen Schliffe untersuchte Formenwelt, welche nach unserer vorl"aufigen "ubersicht und Sch"atzung wohl "uber f"unfzig verschiedenen Arten von Lebewesen angeh"oren mag, von denen aber, da sie meist nur in Struktur- und Bruchst"ucken erhalten sind, nur eine Minderzahl genau zu beschreiben sein wird, scheint einer fr"uhen Entwickelung der Lebewelt auf dem betreffenden Himmelsk"orper anzugeh"oren, vielleicht einer noch fr"uheren als die unserer "altesten Fossilien f"uhrenden Schichten der Erde.
\item Die ganze Tierwelt dieser Meteorite macht zun"achst den Eindruck au"serordentlicher Kleinheit der Formen im Verh"altnis zu den irdischen. Diesen Eindruck erhielt schon Dr. Hahn und auch wir konnten uns demselben zuerst nicht entziehen. In der Tat sind Polypenkelche von 0,04 mm Durchmesser von irdischen Korallen bis jetzt nicht bekannt (doch gibt es von letzteren solche von 0,5 mm Durchm.). Aber wir d"urfen daraus doch wohl noch keinen Schluss ziehen auf die Winzigkeit jener Tierwelt "uberhaupt im Vergleich zur irdischen. Die Gr"o"se der Polycistinenformen, die wir als solche erkannt (und die Hahn als sehr kleine Crinoiden anzusehen geneigt war), sowie der Foraminiferen, stimmt ganz wohl zu den irdischen. "uberdies ist wohl zu bedenken, dass die oft schwer zu deutenden Strukturfetzen und Maschengewebe aller Art, die in den Meteoriten zu Tage treten, recht wohl auch Reste gr"o"serer (aber schwerlich wohl h"oherer) Lebensformen sein k"onnen. Auch im j"ungsten Meereskalk, wie er sich an unseren tropischen Meeresk"usten aus Detritus der Schaltiere, Echinodermen, Korallen, Polythalamien u. s. w. bildet, sind gr"o"sere, besser erhaltene Schalen u. s. f. immer verh"altnism"a"sig selten, w"ahrend mit dem Mikroskop deutbare Strukturreste von solchen h"aufig vorkommen. Dieselben sind aber hier leichter deutbar, weil wir die dazu geh"origen noch lebenden Formen leicht untersuchen k"onnen.
\item Die ganze Formenwelt dieser Meteoriten, soweit wir sie untersuchen konnten, macht den Gesamteindruck einer typisch zusammengeh"origen. Es liegen Schliffe vor von achtzehn verschiedenen Meteorf"allen, zum Teil aus dem vorigen Jahrhundert. Immer kehren dieselben typischen Formen, nur mehr oder weniger h"aufig wieder. Die Annahme scheint uns daher bis auf Weiteres gerechtfertigt, dass alle diese Chondritmeteriten von einem einzigen, au"serirdischen Himmelsk"orper, vielleicht einem geborstenen Planeten herstammen m"ogen, der nach dem analogen Bau seiner Lebeformen wohl auch in seinen physikalischen, besonders aber den atmosph"arischen und W"armeverh"altnissen unserer Erde nicht ganz un"ahnlich gewesen sein kann.
\end{enumerate}
\paragraph{}
Wir wollen es nun versuchen, einige der auffallendsten Gattungen und Arten kurz zu charakterisieren, indem wir uns eine ausf"uhrlichere Beschreibung mit Abbildungen, besonders auch der inneren Strukturverh"altnisse, zu welcher bereits viel Material vorliegt, vorbehalten.
\clearpage
\section{Gittertierchen, Polycystina}
\subsection{\emph{Phormiscus}. Nov. gen.}
\paragraph{}
(φορμισκος = Binsenk"orbchen)%φορμισxος

Facettirte Kugeln, bestehend aus glashellen Kieselnadeln, die wie Binsenk"orbchen in regelm"a"sigen Winkeln "ubereinander gelegt sind. Die Nadeln sind hohl, oft deutlich mit L"ochelchen in L"angsreihen versehen. Hierher:
\subsubsection{\emph{Phormiscus vulgaris}. N. sp.}
\paragraph{}
(Abbildung: Hahn, Meteoriten, Tafel 29, Fig. 2.)

Durchmesser des Ganzen 0,18 mm. Durchmesser der Nadelbalken 0,05 mm. Vom Meteorfall von Knyahinya.

Diese Phormiscusformen sind in Bruchst"ucken au"serordentlich h"aufig in den Meteoriten von Knyahinya. Es gibt verschiedene Arten, die h"aufigste aber ist die obengenannte, welche sofort an den dicken, glashellen, in spitzen Winkeln "ubereinander gekreuzten Nadelb"undeln zu erkennen ist.
\subsubsection{\emph{Phormiscus grandis}. N. sp.}
\paragraph{}
(Abbildung: Hahn, Meteoriten: Tafel 29, Fig. 6.)

Feinmaschiger als die vorige Art. Die Nadeln kreuzen sich unter weit mehr Winkeln.

Die besten, erst nachtr"aglich gefundenen, auch den inneren Bau zeigenden Exemplare sind noch nicht abgebildet. Der Durchmesser eines solchen betr"agt 3,2 mm. Es ist also ein gro"ses, mit blo"sem Auge recht wohl sichtbares Tierchen.

Dass diese \emph{Phormiscus} zu den Polycistinen geh"oren, scheint uns sicher. Die hohlen, teilweise durchl"ocherten Kieselnadeln, besonders aber die Kugelformen, die nur bei frei im Wasser sich bewegenden Tieren denkbar ist, weist zun"achst darauf hin, und nicht auf Schw"amme, an die man sonst auch denken k"onnte. Jedenfalls aber bilden sie eine eigene Familie, die wir Phormiscidae nennen wollen. --- Crinoiden, wie Hahn fr"uher vermutete, sind es sicher nicht.
\subsection{\emph{Thyriscus}. Nov. gen.}
\paragraph{}
(θυρις = Fenster)%θυρις

Gleichfalls facettirte Kugeln, bestehend aus runden Kieselb"allchen, welche in der Art angeordnet sind, dass sie viereckige, nach innen sich verj"ungende Trichter wie Fenster oder noch besser: Schie"sscharten bilden. Die B"allchen sind hohl und mit "ofters deutlichen L"ochelchen versehen. Geh"ort ohne Zweifel auch in die Familie der Phormiscidae.
\subsubsection{\emph{Thyriscus formosus}. N. sp.}
\paragraph{}
(Hahn: Tafel 30, Fig. 3.)

Durchmesser des ganzen, hier abgebildeten Bruchst"ucks 0,70 mm. Durchmesser eines ganzen Trichters 0,35 mm. Durchmesser der einzelnen B"allchen 0,01 mm. Distanz der L"ochelchen von einander 0,006 mm. Durchmesser der L"ochelchen 0,001 mm. Vom Meteorfall von Knyahinya.
\subsection{\emph{Goniobrochus}. Nov. gen.}
\paragraph{}
(γωνια = Winkel, βρὁχος = Masche)%(γωνια = ``cornered'', βρὁχος = ``mesh''.)

Wir begr"unden diese Gattung auf sehr charakteristische Strukturst"ucke, die "ofters in unseren Schliffen vorkommen und von denen Hahn in seinen Meteoriten, Taf. 13, Fig. 6, eines abgebildet hat. Es ist ein fest zusammengef"ugtes, netzartiges Kieselgewebe aus innig verwachsenen, eine zusammenh"angende Scheibe darstellenden Kieselb"allchen gebildet, die sich unter Winkeln kreuzen und fast gleichseitige, viereckige Maschen bilden. Da, wo sich die Leisten kreuzen, entstehen Buckeln wie Kn"opfe eines Netzes. --- Wir k"onnen wohl auch diese Gebilde am ehesten bei den Polycistinen unterbringen, unter denen H"ackel "ahnliche Skelettformen in seinem sch"onen Werke: \emph{Die Radiolarien} Taf. 29 abgebildet hat. Besonders k"amen in Betracht die Gattungen \emph{Stylodictya} und \emph{Stylospira}, die ganz "ahnliche gekn"opfte Netzformen in ihrem inneren Skelett aufweisen. Doch k"onnte man auch an Schw"amme, z. B. an manche \emph{Scyphia} denken; oder an Bryozoen?
\subsubsection{\emph{Goniobrochus haeckelii}. N. sp.}
\paragraph{}
Diese schon von Hahn (siehe oben) abgebildete Form stammt von dem Meteorfall von Cabarras. Das vorliegende St"uck erscheint in dem Schliff f"acherf"ormig ausgebreitet, misst in die Quere 0,5, die H"ohe 0,4 mm. Die Dicke der B"allchen betr"agt 0,01, der Durchmesser einer Masche ebenso 0,01 mm. Das Ganze scheint eine runde Scheibe oder vielleicht auch einen Trichter gebildet zu haben. Wir nennen die Art zu Ehren unseres einstigen Studiengenossen, des ber"uhmten Begr"unders unserer genaueren Kunde von der gro"sen Welt dieser kleinen Organismen.
\clearpage
\section{Schw"amme und Foraminiferen}
\subsection*{Familie: Uranidae. \emph{Nobis}.}
\paragraph{}
Ein sehr charakteristischer Meteoritentypus von niederen Tierformen, der sehr h"aufig in den verschiedensten Meteorf"allen vorkommt und --- wegen der von uns nachtr"aglich aufgefundenen ausgezeichneten Durchschnitte bis jetzt am besten von allen Meteorformen --- kaum die \emph{Hahnia} (s. unten) ausgenommen --- studiert werden konnte. Derselbe l"asst sich an keine der uns bekannten irdischen Tierformen genauer anschlie"sen. Ob Schwamm, ob Foraminifere, diese Frage wird schwer zu entscheiden sein, wie dies ja bekanntlich auch bei manchen fossilen irdischen Formen der Fall ist. Vielleicht haben wir es hier mit einer Mittelform zu tun.

Es sind festsitzende, kissenf"ormige St"ocke mit por"oser und fein lamell"oser Rindenschicht und einem gr"oberen, gleichfalls lamell"osen, Lakunen oder Kammern bildenden inneren Skelett.
\subsection{\emph{Urania}, Hahn (\emph{sensu stricto}).}
\paragraph{}
Wir adoptieren in engerem Sinne den Gattungsnamen von Hahn, den derselbe schon in seinem Werke \emph{Die Urzelle}, allerdings als Pflanzengattung, f"ur eine sehr charakteristische Meteoritenform aufgestellt hat. Ich habe seitdem an einer Reihe von g"unstigen Durchschnitten diese interessanten, in den Meteoriten von Knyahinya besonders h"aufigen Formen studieren und zeichnen k"onnen, so dass ein Zweifel "uber ihre Tiernatur, die auch Hahn sp"ater in seinem Meteoritenwerk angenommen hat, nicht mehr bestehen kann. Es sind immer smalteblaue, kissenf"ormige, wegen der sehr feinen, zart gestrichelten, por"osen Oberhaut samtartig anzusehende, wahrscheinlich auf Stielen festgewachsene St"ocke. Auf dem Querschnitt unterscheidet man sofort eine durchscheinende por"osen Rindenschicht. Das ganze Innere des Kissens besteht aus einem ziemlich unregelm"a"sigen Maschengewebe, in welchem radi"ar von der Rinde nach dem Zentrum zu streichende Lamellenz"uge deutlich werden, welche lakunenartige Hohlr"aume oder Kammern zwischen sich lassen.
\subsubsection{\emph{Urania salve}. N. sp.}
\paragraph{}
So wollen wir sie nennen, denn es war der erste Gru"s einer organischen Form aus einer anderen Welt, das erste Wesen, das Hahn als ein organisches erkannte, wenn auch zuerst als eine Pflanze beschrieb. Diese Art erscheint in gro"sen und kleinen, ganzen Individuen und einer Menge von Bruchst"ucken, sehr h"aufig in den Meteoriten, besonders in denen von Knyahinya. Durchschnittliche Gr"o"se 1 mm. Dicke der au"sen stets smalteblauen Rindenschicht 0,04 mm. Hahn hat sie vielfach abgebildet. Die gro"se Figur von Tafel 2, alle Figuren auf Tafel 3: 1, 2, 3, 4, 5, und 6, sodann Fig. 1, 4 und 6 auf Tafel 4, 1 und 4 auf Tafel 5 geh"oren hierher. Diese Art zeigt oft sehr eigent"umliche, parallele oder radi"ar laufende, tiefe L"angsfalten auf der Oberfl"ache, so dass man an eine gewisse Elastizit"at der Rindenschicht im Leben denken m"ochte.
\subsection{\emph{Pectiscus}. Nov. gen.}
\paragraph{}
(πηκτὁς = gek"ammt)%πηxτὁς

Lappige, wahrscheinlich mit breiter Basis festsitzende St"ocke. Geh"oren zu derselben Familie wie \emph{Urania}, zu den Uranidae. Aber die Rindenschicht ist hier eine andere, gr"obere, wie gek"ammt, d. h. in st"arkeren, mehr oder weniger radi"ar ausstrahlenden Rippen (Lamellen) gebildet, die oft an die Septa gewisser Korallenformen, z. B. der \emph{Fungia}, erinnern. Der innere Bau aber, von dem wir mehrere ganz vortreffliche Durchschnitte vor uns haben (siehe Fig. 1), besteht "ahnlich wie bei \emph{Urania} aus einem lamell"osen, Kammern bildenden Gewebe, das nichts mit der Korallenstruktur zu tun hat. Es gibt eine Reihe von Arten, zum Teil offenbar sehr gro"se, von welchen letzteren aber meist nur das grobe, innere, gekammerte Maschengewebe erhalten ist.
\subsubsection{\emph{Pectiscus zittelii}. N. sp.}
\paragraph{}
Die h"aufigste Art. Erinnert der "au"serlichen Erscheinung nach durch die radi"aren Strahlen und oft auch durch die Gesamtkonturen h"aufig an die bekannten Kammmuscheln (\emph{Pecten}). Doch halten die Lappen dieser St"ocke durchaus keine regelm"a"sige Gesamtform ein. Immer sind sie an den R"andern abgerundet, oft ist der Rand durch seichte Kerben in kleinere Lappen geteilt. Durchmesser der St"ocke von 1 bis 3 mm. Die feinen Rippchen auf der grauen Rinde sind durchschnittlich 0,04 mm von einander entfernt.

Sehr h"aufig in den Meteoriten, besonders in denen von Knyahinya, auch von Siena. Auch das gro"se Gebilde, an das unsere \emph{Hahnia} (siehe unten) wie angeklebt erscheint, ist ein solcher \emph{Pectiscus}.
\begin{figure}[H]
\centering
\includegraphics[scale=1,keepaspectratio]{fig1.png}
\caption{80-mal vergr"ossert}
\end{figure}
\paragraph{}
In Fig. 1 haben wir ein kleines Exemplar abgebildet. Es stammt von dem Meteorfall von Iowa und liefert zugleich ein deutliches Bild der inneren Struktur. Oben und unten ist die hier grau gef"arbte "au"sere Rinde des Stockes erhalten. In der Mitte hat der Schliff dieselbe rasiert und zwar ungleich auf den beiden Seiten; rechts tiefer, daher man dort die vom Boden hereinragenden Lamellen noch ziemlich parallel liegen sieht. In der linken H"alfte dagegen ist der Schliff gerade durch die innerste, unregelm"a"sigere, lakun"ose Mittellage des Lappens durchgegangen. Das ganze St"ockchen ist 1,6 mm lang, 1,2 mm breit. --- Einen "ahnlichen, ebenso instruktiven Durchschnitt haben wir von Knyahinya.

Wir erlauben uns, die Art nach Herrn Professor [Karl Alfred von] Zittel, dem gr"undlichen Erforscher der fossilen Schw"amme, zu benennen.
\subsubsection{\emph{Pectiscus rudis}. N. sp.}
\paragraph{}
Eine kleinere Form mit noch gr"oberen Leisten.
\subsection{\emph{Callaion}. Nov. gen.}
\paragraph{}
(κάλλαιον = Hahnenkamm)%xάλλαιον

Eines der auffallendsten und sch"onsten Gebilde in unserer Meteoritenfauna. Eine feine, wie manche Hahnenk"amme gebuchtete Form, die in ihrem auffallenden Habitus an manche Korallen (\emph{Fungia}, \emph{Herpetolithus}) erinnert, aber nach dem mikroskopischen Bau ihrer Rindenschichte doch wohl auch zu den Uraniden geh"ort. Die d"unne, "au"serste Rindenschicht ist eben so zart bl"aulichgrau, samtartig, dabei aufs Feinste gestreift, wie bei \emph{Urania}. Auf dem Durchschnitt der erhabenen K"amme, die die Buchten des Stockes von einander scheiden, und ebenso an einem feinen L"angsschliff erkennt man das n"achsttiefere, unter der grauen Rinde liegende Gewebe als aus lauter parallelen oder etwas strahlig auseinander laufenden, sehr regelm"a"sigen Lamellen zusammengesetzt, die durch schiefe Br"ucken mit einander verbunden sind Der innerste Bau, wie wir ihn von \emph{Urania} und \emph{Pectiscus} kennen, tritt leider an dem besterhaltenen Unikum nirgends zu Tage, da der Schliff nirgends tief genug eingedrungen. --- Wir wurden bei dieser Form aufs Lebhafteste an den Querschliff von \emph{Carpenteria rhaphidodendron}, [Karl August] M"obius, einer Foraminifere von Mauritius, erinnert, den derselbe in seiner sch"onen Abhandlung "uber das \emph{Eozoon Canadense} (\emph{Palaeontogr. XXV}, Taf. 40 Fig. 60) gegeben.
\subsubsection{\emph{Callaion paulinianum}. N. sp.}
\paragraph{}
Ist in Hahn's Meteoritenatlas noch nicht abgebildet.

Gro"ser Durchmesser des St"ockchens 2,8 mm, der kleinere 2 mm.

Derselbe stellt sich schon dem blo"sen Auge als ein graues, marmoriertes Fleckchen dar. Die parallelen Lamellen, die als zarte Streifen auf der bl"aulichen Oberfl"ache erscheinen, sind 0,002 mm von einander entfernt. Die Lamellen der n"achsten, gr"oberen Schicht 0,01 mm. Die einzelnen Buchten innerhalb des Stockes erscheinen bald als l"angliche Talrinnen von 0,06 mm Durchmesser, bald als rundliche, oder mehr oder weniger eckige, kraterartige Vertiefungen von 0,05 bis bis 0,3 mm Durchmesser. Zwischen diesen Tiefen verlaufen K"amme ganz wie bei \emph{Manicina areolata} und vielen anderen Korallen, aber von wechselnder Breite, 0,05 bis 0,2 mm Durchmesser.

Der Schliff stammt vom Meteorfall von Iowa (Febr. 1847). Leider ist nur ein Exemplar gut erhalten, doch begegneten uns auch in den Meteoriten von Knyahinya "ofters Rudera dieser Art.

Wir erlauben uns, diese Art zu benennen zu Ehren von Fr"aulein Pauline Schloz, der verdienten Schw"agerin des Herrn Dr. Hahn, welche denselben bei der schwierigen Herstellung der vielen Meteoritenschliffe mit aufopferndster Hingebung unterst"utzt hat.
\subsection{\emph{Glossiscus}. Nov. gen.}
\paragraph{}
(γλῶσσα = Zunge)%γλῶσσα

Abgerundete, zungen"ahnliche Lappen bildende St"ocke. Die Oberhaut aus sechseckigen Tafeln zusammengesetzt. Poren in vertieften Furchen und rundlichen, vertieften Nestern; keine Spur von Strahlenrippen wie bei den Uraniden. Ohne Zweifel zu den Schw"ammen geh"orig.
\subsubsection{\emph{Glossiscus schmidtii}. N. sp.}
\paragraph{}
Von Hahn noch nicht abgebildet. An dem vorliegenden, auffallend milchwei"s gef"arbten St"ockchen erscheinen die Poren und Porennester schwarz get"upfelt, indem sich schwarze, organische Masse in den Poren festgesetzt hat, wie dies auch sonst sehr h"aufig in diesen Meteoritenversteinerungen vorkommt. Die ganze L"ange des Lappens betr"agt 1,7 mm, der Querdurchmesser 0,8, Durchmesser der Porennester 0,03 bis 0,05, der Porenfarchen 0,02 bis 0,04 mm, der sechseckigen T"afelchen 0,02 mm.

In einem Schliff von Knyahinya.

Wir erlauben uns die Art zu Ehren des ber"uhmten Erforschers der lebenden Schw"amme, Herrn Prof. Oscar Schmidt in Strassburg, zu benennen.
\subsection{\emph{Carydion}. Nov. gen.}
\paragraph{}
(κάρυον = Nuss)%xάρυον

Glashell durchsichtige, wie die meisten dieser Organismen, in Kieselerde versteinerte Gebilde, die auf dem Durchschnitt ganz einer Nuss mit dicker Schale und Kammern im Inneren gleichen. Die Kammern sind durch ein dickes Balkenwerk hervorgebracht, die dicke Schale ist sehr por"os.

Diese von Hahn nicht abgebildeten Formen sind ziemlich h"aufig in den Meteoriten; wahrscheinlich sind es schwamm"ahnliche Gebilde. Wir wollen nur eine Art beschreiben, deren Bild wir sp"ater geben werden.
\subsubsection{\emph{Carydion solidum}. N. sp.}
\paragraph{}
Durchmesser des Ganzen 0,32 mm. Die L"ochelchen, d. h. Kan"alchen in der Schale haben 0,01 bis 0,005 mm Durchmesser. Die Dicke der das Innenger"uste bildenden Balken betr"agt 0,02 bis 0,05 mm. Die durch die Balken entstehenden Maschen erscheinen drei- oder viereckig. Die Dicke der Rinde oder Schale ist 0,09 mm; die "au"sere Kontur des Ganzen rundlich eckig; die Hohlr"aume sind meist mit schwarzer, organischer Masse ausgef"ullt. Auch die Poren der Rinde sind schwarz tingiert. Die feinere Struktur der Rinde zeigt bei starker Vergr"o"serung rundliche Zellen. --- In einem Schliff von dem Meteorfall von Cabarras.
\subsection{\emph{Brochosphaera}. Nov. gen.}
\paragraph{}
(βρόχος = Masche, and σφαῖρα = Kugel)%(βρόχος = Masche, and σφαῖρα = Kugel)

Sehr h"aufig in den Meteoriten, besonders in denen von Knyahinya, finden sich ziemlich ausgedehnte, gro"smaschige Netze, deren breite Faden aus mehr oder weniger deutlichen, meist sechseckigen Zellen zusammengesetzt sind. Den F"aden entlang h"angen h"aufig schwarze Partikelchen verkohlter, organischer Substanz an. In der Regel sind diese Netze nur in Fetzen erhalten und es war lange unm"oglich, eine Vorstellung von einem Ganzen zu bekommen, endlich aber begegnete mir in einem Knyahinyaschliff ein Gebilde, das einige Aufkl"arung zu geben scheint. Es ist dies eine gro"se, mit blo"sem Auge schon leicht sichtbare, angeschliffene Halbkugel, deren "au"sere Konturen im Wesentlichen erhalten sind und deren Inneres nun eben auf's Sch"onste ein solches Maschenwerk, wie wir es oben beschrieben, enth"alt. Der ganze Rand der Halbkugel, soweit er von dem Schliff nicht getroffen worden, besteht aus lauter ziemlich gleich gro"sen, sechseckigen Zellen oder Pl"attchen. Der innere Raum der Halbkugel, der durch den Schliff blo"sgelegt worden, ist durchzogen von einem vielmaschigen Netz, dessen F"aden aus eben solchen Zellen bestehen, wie jene "au"seren.

Wir k"onnen dieses Gebilde kaum in einer anderen, unserer bekannten Tiergruppen unterbringen, als etwa in der der Schw"amme, aber auch hier w"urde es einen ganz neuen Typus begr"unden. --- Keine dieser Formen ist von Hahn abgebildet.
\subsubsection{\emph{Brochosphaera grandis}. N. sp.}
\paragraph{}
So wollen wir jene Art nennen, von der das bis jetzt besterhaltene St"uck, jene gro"se Halbkugel, vorliegt. Der Durchmesser der ganzen Kugel betr"agt 3,20 mm. Der Durchmesser der Maschen im Inneren 0,2 bis 0,4 mm. Der Durchmesser der oft l"anglichen, oft aber auch ziemlich gleichseitigen, sechseckigen Zellen oder Pl"attchen, die das Ganze zusammensetzen, betr"agt 0,03 bis 0,05 mm. Die durch die dicken F"aden gebildeten rundlichen Maschenr"aume sind in dem vorliegenden Petrefakt mit einem durchsichtig glasigen, vielfach mit feinen Risslinien durchsetzten Silikat ausgef"ullt.

Stammt von dem Fall von Knyahinya.
\subsubsection{\emph{Brochosphaera hexagonalis}. N. sp.}
\paragraph{}
Bei dieser zweiten Art sind die genannten Maschenr"aume konstant sechseckig, sie liegen in dem Netze wie gro"se Kristalle. Ein St"uck dieser Art, von dem auch die "au"seren Konturen ziemlich gut erhalten, misst im Durchmesser 1,20 mm. Die mit Silikat ausgef"ullten sechseckigen, selten f"unfeckigen, kristall"ahnlichen Maschen messen 0,2 mm; die Zellen oder Pl"attchen, die das Netzwerk zusammensetzen, 0,03 bis 0,04 mm.

Stammt von Knyahinya. Auch in einem Pr"aparat von Cabarras findet sich ein sehr "ahnliches. In einem anderen Pr"aparat von Knyahinya erscheinen die gro"sen, sechseckigen Maschen regelm"a"sig in zwei Formen, in gro"sen von 0,26 mm und in kleineren von 0,4 bis 0,3 mm Durchmesser.
\subsection{\emph{Dicheliscus}. Nov. gen.}
\paragraph{}
(διχηλος = mit gespaltenen Klauen)%διχηλος

Eine auffallende und charakteristische Form, bestehend aus einer zusammenh"angenden Traube oder Scheibe von rundlichen Blasen. Der mehr oder weniger stark eingedrungene Einschliff in dieselben gestattet bei manchen eine deutliche Einsicht in das hohle Innere. Man sieht dann ein senkrechtes Diaphragma mitten durch die Blase gehen. Diese Scheidenwand ist immer an der einen Seite dicker als an der anderen; sie entspringt mit breiter Basis von dem Ende der herzf"ormigen Blase und geht lamellenartig sich verd"unnend bis zum anderen Ende. Eine solche angeschliffene Blase mit ihrem Diaphragma gibt etwa das Bild eines zwiegespaltenen Hufs, daher unser Name: \emph{Dicheliscus}. Dass die Blasen unter sich kommunizieren, scheint aus mehreren Stellen des Pr"aparats deutlich, wie wir solche sp"ater abbilden werden.

Wir m"ochten bis auf Weiteres diese Gebilde am Ersten zu den Foraminiferen stellen.
\subsubsection{\emph{Dicheliscus uva}. N. sp.}
\paragraph{}
Ist von Hahn noch nicht abgebildet. Der Durchmesser des ganzen Stocks betr"agt 1,2 mm. L"ange der gr"o"sten, angeschnittenen Blase 0,15 mm. Dicke der Scheidewand 0,01. Die Blasen in dem vorliegenden Pr"aparate sind von verschiedener Gr"o"se und alle "uberg"ange des Anschliffs werden daran deutlich.

Vom Meteorfall von Knyahinya.
\subsection{Weitere Formen}
\paragraph{}
Kleine Bruchst"uckchen von regelm"a"sig gewundenen Gebilden mit polythalamienartigen Kammern, die vielleicht zu diesen Rhizopoden geh"oren, sind uns bei der Durchsicht der Meteorschliffe hin und wieder begegnet. Aber ihre Erhaltung ist meist keine g"unstige. Ein sehr h"ubsches solches St"uckchen, wie ein kleiner Nautilus, steckt in einem Meteoritenschliff von Cabarras. Der ganze Durchmesser des Sch"alchens w"are etwa 0,5 mm, der Kammern 0,05 bis 0,1 mm. Doch bed"urfen diese Formen weiterer Pr"ufung, ehe wir sie festzustellen wagen.
\clearpage
\section{Korallen}
\subsection{\emph{Hahnia}. Nov. gen.}
\paragraph{}
Dies ist die Form, die mich nach den st"arksten Zweifeln zuerst dazu bestimmte, ein genaueres, zoologisches Studium der von Hahn entdeckten Gebilde vorzunehmen. Sie allein w"are auch in der Tat schon entscheidend. Freilich gen"ugen auch hier die photographischen Bilder Hahn's in seinem Meteoritenwerk Taf. 1, 5 und Taf. 10 Fig. 3 und 4 bei weitem nicht. Eine gelbe Eisenf"arbung, die auf dem Pr"aparate liegt, verursachte sehr st"orende, schwarze Schatten und "uberhaupt ist die mikroskopische Photographie noch nicht so weit gelangt, die Bilder mit der Sch"arfe wiederzugeben, wie sie sich unserem Auge darstellen. So sch"atzenswert die photographische Abbildung f"ur gr"o"sere Formen ist, wie die sch"onen Korallenwerke von Dr. Klunzinger und Rominger beweisen, so wird doch bis auf Weiteres f"ur die mikroskopische Darstellung die mit vollem Verst"andnis zeichnende Hand der Forschers selbst durch jene mechanische Darstellung noch nicht, vielleicht niemals ersetzt. Unsere \emph{Hahnia}, Fig. 2, ist leider bis jetzt ein Unikum geblieben. Der betreffende Schliff geh"ort zum Meteorfall von Knyahinya. Er ist einer der gl"ucklichsten und enth"alt au"serdem noch sehr gute \emph{Urania}-, \emph{Pectiscus}- und \emph{Phormiscus}- Reste.

Gattungscharaktere von \emph{Hahnia}: Polypenr"ohren mikroskopisch klein, ungleich, gro"se mit kleinen gemischt, mehr oder weniger polygonal mit abgerundeten Ecken. Die W"ande der R"ohren dick, mit scharfer, line"arer Begrenzung nach au"sen. Bei st"arkerer Vergr"o"serung wird zwischen den die benachbarten Polypen begrenzenden Linien ein gleichm"a"sig dickes Zwischenr"ohrengewebe (C"onenchym) sichtbar, welches auf dem Querschliff ein deutliches Netzwerk darstellt. Innere L"angsleisten (Septa) in den R"ohren fehlen, ebenso die Querscheidew"ande (Tabulae), welche letztere bekanntlich bei vielen "ahnlichen, irdischen Korallen die einzelnen R"ohren in Etagen "ubereinander teilen. Stock wahrscheinlich inkrustierend, flach, kuchenf"ormig.

Die Gattung geh"ort wahrscheinlich zu den Favositidae, einer Korallenfamilie, die auf der Erde l"angst ausgestorben, in der Silur- und Devon- Formation ihre Bl"ute gehabt hat und von der eine gro"se Zahl von sehr verschiedenen Formen, die aber noch einer weiteren zoologischen Sichtung bed"urfen, in der Pal"aontologie beschrieben ist.
\begin{figure}[H]
\centering
\includegraphics[scale=1,keepaspectratio]{fig2.png}
\caption{\emph{Hahnia meteoritica}, N., an einem \emph{Pectiscus} sitzend. 80-mal vergr"ossert.}
\end{figure}
\paragraph{}
Durchmesser des ganzen St"ockchens 0,90 mm, also eben noch mit dem blo"sen Auge als eine kleine Linse im Schliff zu erkennen. Durchmesser der einzelnen Polypenkelche 0,04 bis 0,1 mm. Durchmesser der gelben Zwischenstra"sen, des C"onenchyms, 0,008 mm. An den Ecken schwillt dasselbe, wie h"aufig bei den \emph{Favositen}, etwas st"arker an. Die frappante "ahnlichkeit dieses St"ockchens mit \emph{Favosites polymorphus} aus dem Devon ist schon Prof. Quenstedt aufgefallen, als ihm Dr. Hahn das Objekt zeigte. Noch mehr ist es mit \emph{Favosites bimuratus} aus dem Devon von Bensberg zu vergleichen, wo die Polypenw"ande und das C"onenchym au"serordentlich "ahnlich sich darstellen, freilich immer mit Ausnahme der Gr"o"senverh"altnisse. Denn bei \emph{Favosites bimuratus} messen die Kelche immer noch ein halb bis 1 mm.

Die einzelnen Polypenkelche bei unserer \emph{Hahnia} sind mit einer schwarzgrauen Masse gef"ullt, die W"ande erscheinen grauwei"s, das C"onenchym gelb. Durch einen gl"ucklichen Zufall wurde dieser Korallenstock gerade von oben getroffen. In der Mitte des Bildes erscheinen die Kelche fast intakt; rings am Rande herum, besonders auf der linken Seite, sind sie etwas verschliffen, so dass man dort f"ur die Strukturerkenntnis sehr wertvolle Halbl"angsschliffe durch die Polypenr"ohren erh"alt und den Mangel von Querscheidenw"anden, sowie auch von Gef"assl"ochelchen (Sprossenkan"alen) konstatieren kann.

Hahn's Abbildung Taf. 1 f. 5 und Taf. 10 f. 4 ist leider durch die gelbe F"arbung des Pr"aparats, die in der Photographie schwarz kommt, beeintr"achtigt.
\subsection{\emph{Calamiscus}. Nov. gen.}
\paragraph{}
(καλαμίσκς = R"ohrchen)%xαλαμίσxς

Favositenartige Polypenst"ocke, bestehend aus regelm"a"sig parallel nebeneinander oder auch etwas radi"ar verlaufenden, meist glashell durchsichtigen R"ohren ohne L"angsleisten (Septa) im Innern, aber mehr oder weniger regelm"a"sig durch Querscheidew"ande oder B"oden (Tabulae) in Etagen geteilt und oft sehr regelm"a"sig mit feinen L"ochelchen, die die Gef"asscommunication zwischen den benachbarten R"ohren vermitteln, ausgestattet. Diese vollkommene "ubereinstimmung der Struktur mit der vieler fossiler Favositkorallen aus der Devon- und Silur- Formation der Erde l"asst uns trotz der Kleinheit der vorliegenden meteoritischen Formen nicht wohl an etwas Anderes denken, als an Korallenpolypen. Leider sind fast nur Seitenschliffe erhalten, weil in dieser Richtung die Polypenst"ocke am leichtesten zerbrechen. Bei dem Mangel befriedigender Querschliffe wird eine Unterscheidung der Arten von Calamiscus sehr schwer; es bleibt dazu fast nur "ubrig: die durchg"angige Weite der Polypenr"ohren, die Distanz der B"oden und der Gef"a"sl"ocher von einander, die waagerechte oder schiefe Richtung der B"oden u. dgl., lauter Merkmale, die auch bei einer und derselben Art schon ziemlich variieren. --- Diese Gebilde sind au"serordentlich h"aufig in den Meteoriten, besonders in denen von Knyahinya.
\subsubsection{\emph{Calamiscus g"umbelii}. N. sp.}
\paragraph{}
(Abbildung: Hahn, Meteoriten: Tafel 14 und 15.)

Wir st"utzen diese Art auf eines der besterhaltenen St"ockchen in einem Meteoritenschliff von dem Fall von Cabarras. Es ist ein l"anglicher, nach unten spitz verlaufender Stock, wie Favositenst"ocke es wegen der Art ihrer Vermehrung durch Zwischenschiebung unten spitziger, neuer R"ohren gew"ohnlich sind. Das vorliegende St"ockchen hat einen Durchmesser von 0,46 mm und eine H"ohe von 1 mm, ist also recht wohl noch mit blo"sem Auge sichtbar. Der Durchmesser der R"ohren betr"agt 0,01 mm, die Distanz der an diesem Polypenst"ockchen au"serordentlich sch"on sichtbaren Gef"assl"ochelchen von einander 0,005 bis 0,01 mm. Die s"agen"ahnliche Auskerbung an der Seite einer R"ohre in dem Hahn'schen Bilde entstand durch eine zuf"allige Anschleifung derselben, in der Art, dass die trichterf"ormige Einbuchtung der L"ochelchen zu Tage tritt. Die B"oden stehen etwas schief in der R"ohre sehr unregelm"a"sig weit von einander und sind "uberhaupt in diesem Stock seltener als in manchen anderen.

Wir erlauben uns, diese Art nach Herrn Director G"umbel in M"unchen zu benennen, der die Chondrit-Meteoriten zuerst einer genauen mikroskopischen Untersuchung unterzogen und bei seiner trefflichen Beschreibung der Chondren in seiner Abhandlung "uber die in Bayern gefundenen Steinmeteoriten (Sitzungs-Ber. der K. bayer. Akad. d. Wissensch. zu M"unchen 1878, S. 14) wahrscheinlich solche Calamiscusformen, die aber weniger gut erhalten waren, vor sich gehabt, sie aber mineralogisch zu deuten versucht hat.
\subsection{\emph{Bosea}. Nov. gen.}
\paragraph{}
Eines der sch"onsten Meteoritengebilde, ohne Zweifel ein St"uckchen eines Korallenstocks. Ein ziemlicher Teil der Oberfl"ache mit vielen deutlichen, gr"o"seren und kleineren Sternchen ist an dem Unikum erhalten. Die Sternchen bildeten, wie es scheint, erhabene, abgeflachte Kegelchen; sie haben bis zu zehn nach au"sen sich verbreiternde Septa, getrennt durch dunklere Furchen. Die Mitte des Sternchens, von der die Septa und die Furchen ausgehen, besteht aus eckigen K"ornchen. Das Coenenchym oder Zwischenfeld zwischen den Sternchen erscheint mit eckigen Pl"attchen gepflastert. Kleinere, offenbar j"ungere Sternchen mit weniger Strahlen erscheinen zwischen den alten, ganz wie bei einer Astraea.

Ich erlaube mir, die Gattung zu benennen zu Ehren von Herrn [Carl August] Carl Graf von Bose und Frau [Wilhelmine Emilie] Louise Gr"afin von Bose geb. Gr"afin von Reichenbach-Lessonitz, welche beide selbst ausgezeichnete Naturkenner, an diesen Meteoritenstudien des Verfassers den lebhaftesten Anteil genommen. Wie bekannt hat Frau Gr"afin Bose vor nicht langer Zeit durch eine Stiftung in Frankfurt a. M. ihr Interesse f"ur die Erforschung der Natur in gro"sartigster Weise bet"atigt.
\subsubsection{\emph{Bosea cyanea}. Nov. Sp.}
\paragraph{}
Das genannte St"ockchen, am Rande "uberall abgebrochen, hat, soweit erhalten, eine L"ange von 1,44 mm, eine Breite von 0,88 mm. Der Durchmesser der Sternchen betr"agt 0,04 bis 0,08 mm. Der Durchmesser der vertieften, von der Mitte ausstrahlenden Furchen ist 0,003 bis 0,006 mm. --- Das Versteinerungsmaterial zeigt dieselbe smalteblaue Farbe, wie bei \emph{Urania salve}. --- Dieses Unikum steckt in einem Schliff vom Fall von Knyahinya.
\clearpage
\section{\emph{Crinoidea}}
\paragraph{}
Unser verehrter Freund, Dr. Hahn, hat in seinem Meteoritenwerke, Taf. 16 - 30, eine gr"o"sere Anzahl von Formen vorl"aufig zu dieser niedersten Klasse der Echinodermen stellen zu m"ussen geglaubt. Wir haben nach genauerem Studium ihrer Organisation, soweit diese zu entziffern, eine Reihe derselben den Polycistinen und Schw"ammen, beziehungsweise Foraminiferen verwandter gefunden. Doch bleibt eine Anzahl Formen "ubrig, die wir vorl"aufig der obigen Tierklasse zuz"ahlen wollen, da sie ohne Zwang keinem anderen uns bekannten Tiertypus anzureihen sind und immerhin gewisse Strukturmerkmale mit den Crinoiden gemein haben.
\subsection{\emph{Eulophiscus}. Nov. gen.}
\paragraph{}
(εὔλοφος = mit sch"onem Busch)%εὔλοφος

Ein B"uschel f"acherf"ormig von einem Mittelpunkt ausstrahlender, im Leben ohne Zweifel frei flottierender, unten, nahe dem Ursprung ein- oder zweimal gegabelter, weiterhin aber nicht mehr verzweigter, ziemlich gleich dicker Arme.
\subsubsection{\emph{Eulophiscus quenstedtii}. N. sp.}
\paragraph{}
Wir beziehen hierher in erster Linie das h"ubsche Bild, das Hahn als Titel auf den Umschlag seines Meteoritenwerkes gew"ahlt und kleiner auf Taf. 22 Fig. 3 abgebildet hat. Auch dieses Objekt gew"ahrt aber unter dem Mikroskop ein viel deutlicheres Bild, als die Photographie leistete. Wir sehen von einer Basis aus zun"achst f"unf dickere Arme ausgehen; der linke "au"serste, am g"unstigsten gelegene, zeigt unten einen Querschnitt von 0,04 mm. Schon 0,08 mm "uber seinem Ursprung gabelt sich derselbe aufs Sch"onste in zwei Hauptarme von 0,02 mm Dicke. So bleiben sie sich dann gleich, soweit man sie verfolgen kann, was bei dem linksliegenden bis ans Ende der F"achers, soweit dieser erhalten, m"oglich ist. Die genannte Gabelung hat ganz die Form, wie wir sie bei den Crinoiden gewohnt sind. Doch ist weder hier noch bei den "ubrigen Armen eine deutliche Quergliederung sichtbar. Dass diese Arme im Leben frei im Wasser flottierten, ist sicher anzunehmen, denn man sieht sie an mehreren Stellen sich "uber einander legen und kreuzen, unter einander verstecken u. s. f. Die Gr"o"se des ganzen B"uschels ist freilich f"ur einen Crinoiden sehr unbedeutend; die H"ohe des ganzen B"uschelchens betr"agt nur 0,7, die Breite 1 mm. Das Ganze erscheint graulich gef"arbt, die genannten Hauptarme gelblich, halbdurchsichtig.

Stammt von Fall von Knyahinya.

Hierher vielleicht auch noch die Formen: Hahn, Meteoriten Taf. 22, Fig. 5 und 6.
\subsection{\emph{Euplocamus}. Nov. gen.}
\paragraph{}
(εὐπλοκαμος = sch"onhaarig)%εὐπλοκαμος

Eine der vorigen "ahnliche Gattung, bei welcher aber die Arme nicht gegabelt sind.
\subsubsection{\emph{Euplocamus algoideus}. N. sp.}
\paragraph{}
Diese Gattung und Art st"utzen wir zun"achst auf das Hahn'sche Bild, Taf. 1 Fig. 6, Taf. 25 Fig. 1 und Taf. 19, welche alle dasselbe Objekt darstellen, und diese Bilder kann man als ziemlich gelungen bezeichnen. Das h"ubsche St"uckchen selbst macht unter dem Mikroskop ganz den Eindruck eines B"uschelchens Seealgen, die an einem Felsst"uck festgewachsen. Von einer pflasterf"ormig gebauten Zentralscheibe aus strahlt hier, b"uschelf"ormig wie bei den vorigen, eine gro"se Anzahl gleich dicker Arme aus, die sich, so weit erhalten, nicht verj"ungen. Der Durchmesser der Arme betr"agt 0,04 mm. Die Arme sind glashell durchsichtig. Durch das Innere eines jeden derselben l"auft eine dunkle Kontur, die auf einen feinen Hohlraum schlie"sen l"asst. Auch hier legen und schieben sich die Arme durch- und "uber- einander, so dass man notwendig an ein einstiges, freies Flottieren derselben denken muss. Das ganze St"ockchen hat eine H"ohe von 0,8 mm und eine Breite von 1,1 mm, ist also wie das vorige recht wohl noch mit blo"sem Auge sichtbar.

Stammt vom Meteorfall von Knyahinya.
\subsubsection{\emph{Euplocamus articulatus}. N. Sp.}
\paragraph{}
(Abbildung: Hahn, Tafel 23: Figur 4.)

Ein sehr h"ubsches und deutliches, in dem photographischen Bilde aber weniger gelungenes Objekt. Aus einer von vielen kleinen, eckigen Pl"attchen gebildeten Basis entspringt eine Quaste von zun"achst scheinbar ungegliederten, runden, stabf"ormigen, weiter oben durch deutliche Gliederung ausgezeichneten Armen. Die Gliederung derselben beginnt in dem Objekte bei einer sehr markierten Knickung der Arme. Diese haben, wie aus dem vorliegenden Petrefakt sicher hervorgeht, frei durch und "uber einander flottiert. Die einzelnen Arme sind rund, ein innerer Hohlraum ist nicht sichtbar, daher es wohl sp"ater von der Gattung Euplocamus wird getrennt werden m"ussen. Der Durchmesser des Ganzen betr"agt 1,60 mm. Der Durchmesser der Arme unter dem Knie 0,08 mm. Nach oben verj"ungen sie sich etwas, aber wenig. Der Durchmesser der eckigen Pl"attchen der Basis ist 0,03 bis 0,04 mm. Die Farbe des Ganzen ist gelblich, sch"on metallisch gl"anzend. --- Es steckt in einem Schliff von dem Meteorfall von Knyahinya.
\subsection{\emph{Crobyliscus}. Nov. gen.}
\paragraph{}
(κρὠβυλος = Zopf)%xρὠβυλος

An einen deutlich aus vieleckigen, meist sechseckigen Pl"attchen gebildeten Hohlraum schlie"sen sich oben Anzahl zylindrischer, zopff"ormiger, sich nach dem Ende zu verj"ungender, massiver (nicht hohler), aus eckigen Scheibchen gebildeter, armf"ormiger Anh"ange an. Ist es ein Crinoid und ist jener Hohlraum der Kelch desselben? Das Fragment, auf das wir diese Gattung begr"unden, ist bis jetzt ein Unikum, dessen Bild wir in unser gr"o"seren Abhandlung bringen werden.
\subsubsection{\emph{Crobyliscus fraasii}. N. Sp.}
\paragraph{}
L"angsdurchmesser des Ganzen, soweit erhalten, 0,74 mm. Querdurchmesser des Kelchs 0,45 mm. L"ange der Arme, soweit erhalten, 0,35 mm. Querdurchmesser der Arme 0,3 bis 0,6 mm. Dicke der Wirtel, die die Arme zusammensetzen, 0,01 bis 0,02 mm. Durchmesser der eckigen Pl"attchen, die den Kelch zusammensetzen, 0,03 bis 0,05 mm. Das Mineral, aus dem das Gebilde jetzt besteht, ist zweifelsohne Kieselerde.

Von dem Meteorfall von Knyahinya.
\clearpage
\section{Schlussfolgerung}
\paragraph{}
Mit der vorl"aufigen Charakterisierung obiger sechzehn Gattungen von Meteoritenformen glauben wir, f"ur jetzt wenigstens, die Basis einer kleinen Meteoritenfauna gelegt zu haben. Von allen nicht abgebildeten und au"serdem von vielen schon photographisch von Hahn dargestellten, soweit sie weniger gelungen, werden wir in unserer in Vorbereitung begriffenen, gr"o"seren Abhandlung genaue, selbstgezeichnete Bilder geben. Dieselben liegen bereits meist fertig vor.

Bez"uglich der Nomenklatur bitten wir, bei s"amtlichen oben aufgestellten, neuen Gattungen --- ausgenommen \emph{Hahnia} und \emph{Bosea} --- als Autorit"at unserem eigenen Namen den Namen unseres werten Freundes Dr. Hahn zuzuf"ugen, der, wenn er auch keinen unmittelbaren Anteil an unserer Arbeit genommen, doch immer Derjenige bleibt, der zuerst die organische Herkunft dieser Formen behauptet und durch seinen immer wertvollen Atlas zu begr"unden versucht hat und auf dessen uns so freundlich zur Bearbeitung "uberlassener, reicher Sammlung unsere obige Arbeit beruht.

Da wir diese Untersuchungen mit Eifer fortzusetzen gedenken, erlauben wir uns zum Schlusse noch die freundliche Bitte an etwaige Besitzer von sicher beglaubigten Meteoritenst"ucken oder Schliffen, dieselben uns gef"alligst zur mikroskopischen Untersuchung mitzuteilen. Wir werden dieselben stets in m"oglichster B"alde, unter Mitteilung des Resultats und nachfolgender "offentlicher Danksagung, zur"uckstellen. --- Unsere Adresse ist: Dr. D. F. Weinland, Esslingen, W"urttemberg.

Druck von E. Blochmann und Sohn in Dresden.
\clearpage
\end{document}
